\documentclass{article}
\usepackage[utf8]{inputenc}
\usepackage{amssymb}
\usepackage{hyperref}
\usepackage{cite}
\title{Ordinals}
\author{Paul Crowley}
\begin{document}
\maketitle
Ordinals are cool.

\textbf{Axiom of Extensionality:} In what follows, the only thing that can be an element of a set is another set; this is implicit in ZFC's Axiom of Extensionality. There are other ways of looking at sets but this is the most usual way in mathematical logic.

\textbf{Axiom of Foundation:} I'm also going to make use of the Axiom of Foundation (aka the Axiom of Regularity), which states that every nonempty set has at least one element that's disjoint with the set: $\forall S \neq \emptyset:  \exists x \in S: x \cap S = \emptyset$. This axiom disallows all sorts of pathological sets; for example, we can't have a set $A$ such that $A \in A$, since then the set $\{A\}$ would have no disjoint element. Similarly, we can't have a loop $A \in B \in C \in A$, since the set $\{A, B, C, A\}$ is disallowed; the same goes for an infinite chain $A_1 \ni A_2 \ni A_3 \ni \ldots$ via the set $\{A_1, A_2, A_3, \ldots\}$---though note that $A_1 \in A_2 \in A_3 \in \ldots$ can be allowed, so long as $A_1$ is disjoint with $\{A_1, A_2, A_3, \ldots\}$.

\textbf{Transitive Set:} A transitive set is a set all of whose elements are also subsets; $A$ is a transitive set iff $\forall a \in A: a \subseteq A$.

\textbf{Ordinal:} An ordinal is a transitive set of transitive sets. $\alpha$ is an ordinal iff $\forall \beta \in \alpha: \beta \subseteq \alpha \wedge \forall \gamma \in \beta: \gamma \subseteq \beta$. This isn't the standard definition of an ordinal, but it's much easier to work with; one of the things that motivated me to write this document was how much easier everything got when I tried using this definition instead.

\textbf{0:} The empty set is an ordinal. It doesn't have any elements that aren't subsets, and it doesn't have any that aren't transitive sets! The usual notation for this is $\emptyset$ or $\{\}$, but here we'll call it simply 0.

\textbf{Successor:} If $\alpha$ is an ordinal, so is $\alpha \cup \{\alpha\}$; we call this the successor, or $\alpha + 1$. Proof: Since $\alpha$ is a transitive set, all its elements are subsets of $\alpha$ and therefore also subsets of $\alpha + 1$. By construction $\alpha$ is also a subset of $\alpha + 1$; therefore it's a transitive set. Since both $\alpha$ and all its elements are transitive sets, $\alpha + 1$ consists entirely of transitive sets. Therefore $\alpha + 1$ is an ordinal. This allows us to construct lots of ordinals:

\begin{tabular}{ c c c }
0 & $\{\}$ & $\{\}$ \\
1 & $\{0\}$ & $\{\{\}\}$ \\
2 & $\{0, 1\}$ & $\{\{\}, \{\{\}\}\}$ \\
3 & $\{0, 1, 2\}$ & $\{\{\}, \{\{\}\}, \{\{\}, \{\{\}\}\}\}$ \\
\end{tabular}

In each case we have eg $3 = 2 + 1 = 2 \cup \{2\} = \{0, 1\} \cup \{2\} = \{0, 1, 2\}$.

\textbf{Union:} The union of a set of ordinals is an ordinal. Since all elements of ordinals are transitive sets, all elements of the union will be transitive sets; since all elements are subsets of some component of the union they are subsets of the whole union.

\textbf{$\omega$:} This gives us our first infinite ordinal, $\omega = \bigcup\{0, 1, 2, 3, \ldots\} =  \{0, 1, 2, 3, \ldots\}$. We can also construct $\omega + 1$ in the usual way.

\textbf{Elements of ordinals are ordinals:} An element of an ordinal is also a subset. Since the ordinal is composed of transitive sets, so is any element.

\textbf{Among ordinals, membership is transitive:} If $\alpha_1, \alpha_2, \alpha_3$ are ordinals such that $\alpha_1 \in \alpha_2 \in \alpha_3$ then $\alpha_1 \in \alpha_3$ since $\alpha_2 \in \alpha_3$ implies $\alpha_2 \subseteq \alpha_3$.

\textbf{Set membership is asymmetric:} The Axiom of Foundation forbids sets such that $A \in A$ or $A \in B \in A$; set membership is therefore both irreflexive and antisymmetric.

\textbf{Set membership is well-founded:} The Axiom of Foundation guarantees that any nonempty set $S$ will have at least one element $x$ such that $\forall y \in S: y \not\in x$. We call this an $\in$-minimal element.

\textbf{Within an ordinal, membership is trichotomous:} $\forall \beta_1, \beta_2, \in \alpha: \beta_1 \in \beta_2 \vee \beta_1 = \beta_2 \vee \beta_1 \ni \beta_2$. The price we pay for all our other proofs being so straightforward is that this one is a little tricky: \cite{1057102}

\begin{itemize}
\item we say that $x, y$ are \emph{incomparable} if they don't meet this condition. Suppose $\alpha$ is an ordinal with incomparable elements.
\item Let $x \in \alpha$ be $\in$-minimal among all elements that are incomparable with some other element
\item Let $y \in \alpha$ be $\in$-minimal among all elements that are incomparable with $x$.
\item For any $z \in y$: since $y$ is $\in$-minimal among elements not comparable with $x$ we have that $z \in x \vee z = x \vee z \ni x$. In the last two cases we have $x \in y$, so we must have $z \in x$ ie $y \subseteq x$.
\item For any $z \in x$: since $x$ is $\in$-minimal among elements incomparable with some other element we have that $z \in y \vee z = y \vee z \ni y$. In the last two cases we have $y \in x$, so we must have $z \in y$ ie $x \subseteq y$.
\item Hence $x = y$, contradicting our initial assumption.
\end{itemize}

\textbf{Ordinals are trichotomous:} For any two ordinals $\alpha_1, \alpha_2$, $\alpha_1 \in \alpha_2 \vee \alpha_1 = \alpha_2 \vee \alpha_1 \ni \alpha_2$. Proof: both are elements of the ordinal $(\alpha_1 + 1) \cup (\alpha_2 + 1)$ and any two elements of the same ordinal are trichotomous.

\textbf{Ordinals are well-ordered by membership:} this just puts together what we've already proven. This also means that the elements of an ordinal are well-ordered by membership, as required by the standard Von Neumann definition.

\bibliography{bib}{}
\bibliographystyle{alpha}
\end{document}
